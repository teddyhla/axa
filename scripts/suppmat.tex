% Options for packages loaded elsewhere
\PassOptionsToPackage{unicode}{hyperref}
\PassOptionsToPackage{hyphens}{url}
%
\documentclass[
]{article}
\usepackage{amsmath,amssymb}
\usepackage{lmodern}
\usepackage{iftex}
\ifPDFTeX
  \usepackage[T1]{fontenc}
  \usepackage[utf8]{inputenc}
  \usepackage{textcomp} % provide euro and other symbols
\else % if luatex or xetex
  \usepackage{unicode-math}
  \defaultfontfeatures{Scale=MatchLowercase}
  \defaultfontfeatures[\rmfamily]{Ligatures=TeX,Scale=1}
  \setmainfont[]{Source Sans Pro}
\fi
% Use upquote if available, for straight quotes in verbatim environments
\IfFileExists{upquote.sty}{\usepackage{upquote}}{}
\IfFileExists{microtype.sty}{% use microtype if available
  \usepackage[]{microtype}
  \UseMicrotypeSet[protrusion]{basicmath} % disable protrusion for tt fonts
}{}
\makeatletter
\@ifundefined{KOMAClassName}{% if non-KOMA class
  \IfFileExists{parskip.sty}{%
    \usepackage{parskip}
  }{% else
    \setlength{\parindent}{0pt}
    \setlength{\parskip}{6pt plus 2pt minus 1pt}}
}{% if KOMA class
  \KOMAoptions{parskip=half}}
\makeatother
\usepackage{xcolor}
\usepackage[margin=1in]{geometry}
\usepackage{color}
\usepackage{fancyvrb}
\newcommand{\VerbBar}{|}
\newcommand{\VERB}{\Verb[commandchars=\\\{\}]}
\DefineVerbatimEnvironment{Highlighting}{Verbatim}{commandchars=\\\{\}}
% Add ',fontsize=\small' for more characters per line
\usepackage{framed}
\definecolor{shadecolor}{RGB}{248,248,248}
\newenvironment{Shaded}{\begin{snugshade}}{\end{snugshade}}
\newcommand{\AlertTok}[1]{\textcolor[rgb]{0.94,0.16,0.16}{#1}}
\newcommand{\AnnotationTok}[1]{\textcolor[rgb]{0.56,0.35,0.01}{\textbf{\textit{#1}}}}
\newcommand{\AttributeTok}[1]{\textcolor[rgb]{0.77,0.63,0.00}{#1}}
\newcommand{\BaseNTok}[1]{\textcolor[rgb]{0.00,0.00,0.81}{#1}}
\newcommand{\BuiltInTok}[1]{#1}
\newcommand{\CharTok}[1]{\textcolor[rgb]{0.31,0.60,0.02}{#1}}
\newcommand{\CommentTok}[1]{\textcolor[rgb]{0.56,0.35,0.01}{\textit{#1}}}
\newcommand{\CommentVarTok}[1]{\textcolor[rgb]{0.56,0.35,0.01}{\textbf{\textit{#1}}}}
\newcommand{\ConstantTok}[1]{\textcolor[rgb]{0.00,0.00,0.00}{#1}}
\newcommand{\ControlFlowTok}[1]{\textcolor[rgb]{0.13,0.29,0.53}{\textbf{#1}}}
\newcommand{\DataTypeTok}[1]{\textcolor[rgb]{0.13,0.29,0.53}{#1}}
\newcommand{\DecValTok}[1]{\textcolor[rgb]{0.00,0.00,0.81}{#1}}
\newcommand{\DocumentationTok}[1]{\textcolor[rgb]{0.56,0.35,0.01}{\textbf{\textit{#1}}}}
\newcommand{\ErrorTok}[1]{\textcolor[rgb]{0.64,0.00,0.00}{\textbf{#1}}}
\newcommand{\ExtensionTok}[1]{#1}
\newcommand{\FloatTok}[1]{\textcolor[rgb]{0.00,0.00,0.81}{#1}}
\newcommand{\FunctionTok}[1]{\textcolor[rgb]{0.00,0.00,0.00}{#1}}
\newcommand{\ImportTok}[1]{#1}
\newcommand{\InformationTok}[1]{\textcolor[rgb]{0.56,0.35,0.01}{\textbf{\textit{#1}}}}
\newcommand{\KeywordTok}[1]{\textcolor[rgb]{0.13,0.29,0.53}{\textbf{#1}}}
\newcommand{\NormalTok}[1]{#1}
\newcommand{\OperatorTok}[1]{\textcolor[rgb]{0.81,0.36,0.00}{\textbf{#1}}}
\newcommand{\OtherTok}[1]{\textcolor[rgb]{0.56,0.35,0.01}{#1}}
\newcommand{\PreprocessorTok}[1]{\textcolor[rgb]{0.56,0.35,0.01}{\textit{#1}}}
\newcommand{\RegionMarkerTok}[1]{#1}
\newcommand{\SpecialCharTok}[1]{\textcolor[rgb]{0.00,0.00,0.00}{#1}}
\newcommand{\SpecialStringTok}[1]{\textcolor[rgb]{0.31,0.60,0.02}{#1}}
\newcommand{\StringTok}[1]{\textcolor[rgb]{0.31,0.60,0.02}{#1}}
\newcommand{\VariableTok}[1]{\textcolor[rgb]{0.00,0.00,0.00}{#1}}
\newcommand{\VerbatimStringTok}[1]{\textcolor[rgb]{0.31,0.60,0.02}{#1}}
\newcommand{\WarningTok}[1]{\textcolor[rgb]{0.56,0.35,0.01}{\textbf{\textit{#1}}}}
\usepackage{longtable,booktabs,array}
\usepackage{calc} % for calculating minipage widths
% Correct order of tables after \paragraph or \subparagraph
\usepackage{etoolbox}
\makeatletter
\patchcmd\longtable{\par}{\if@noskipsec\mbox{}\fi\par}{}{}
\makeatother
% Allow footnotes in longtable head/foot
\IfFileExists{footnotehyper.sty}{\usepackage{footnotehyper}}{\usepackage{footnote}}
\makesavenoteenv{longtable}
\usepackage{graphicx}
\makeatletter
\def\maxwidth{\ifdim\Gin@nat@width>\linewidth\linewidth\else\Gin@nat@width\fi}
\def\maxheight{\ifdim\Gin@nat@height>\textheight\textheight\else\Gin@nat@height\fi}
\makeatother
% Scale images if necessary, so that they will not overflow the page
% margins by default, and it is still possible to overwrite the defaults
% using explicit options in \includegraphics[width, height, ...]{}
\setkeys{Gin}{width=\maxwidth,height=\maxheight,keepaspectratio}
% Set default figure placement to htbp
\makeatletter
\def\fps@figure{htbp}
\makeatother
\setlength{\emergencystretch}{3em} % prevent overfull lines
\providecommand{\tightlist}{%
  \setlength{\itemsep}{0pt}\setlength{\parskip}{0pt}}
\setcounter{secnumdepth}{-\maxdimen} % remove section numbering
\ifLuaTeX
  \usepackage{selnolig}  % disable illegal ligatures
\fi
\IfFileExists{bookmark.sty}{\usepackage{bookmark}}{\usepackage{hyperref}}
\IfFileExists{xurl.sty}{\usepackage{xurl}}{} % add URL line breaks if available
\urlstyle{same} % disable monospaced font for URLs
\hypersetup{
  pdftitle={Supplementary Material for analysis of Anti-Xa and APTTr anticoagulation on VV-ECMO patients},
  pdfauthor={Teddy Tun Win HLA on behalf of the study group},
  hidelinks,
  pdfcreator={LaTeX via pandoc}}

\title{Supplementary Material for analysis of Anti-Xa and APTTr
anticoagulation on VV-ECMO patients}
\author{\href{https://github.com/teddyhla}{Teddy Tun Win HLA} on behalf
of the study group}
\date{`r Sys.Date()'}

\begin{document}
\maketitle

{
\setcounter{tocdepth}{3}
\tableofcontents
}
\hypertarget{objective}{%
\section{1. OBJECTIVE}\label{objective}}

\hypertarget{outcome-variables}{%
\subsection{Outcome variables}\label{outcome-variables}}

\begin{enumerate}
\def\labelenumi{\arabic{enumi}.}
\item
  Bleeding or thrombotic events(BTE) 1a. any BTE 1b. only hemorrhagic
  complications 1c. only thrombotic compications
\item
  ECMO circuit changes
\item
  Heparin Prescription 3a. cumulative dose of heparin 3b. heparin
  prescription changes
\item
  Blood products consumption
\end{enumerate}

\hypertarget{calculated-variables}{%
\subsection{Calculated variables}\label{calculated-variables}}

\begin{enumerate}
\def\labelenumi{\arabic{enumi}.}
\tightlist
\item
  Time in Therapeutic Range (Rosendaal Method)
\item
  Variability in Anticoagulation (Fihn's method)
\end{enumerate}

\hypertarget{section}{%
\section{}\label{section}}

\hypertarget{data-source}{%
\section{3. DATA SOURCE}\label{data-source}}

{[}refer to Barnie's{]}

\begin{itemize}
\tightlist
\item
  Electronic health records data extraction from GSTT systems using ECMO
  pump speed as defintion of cases.
\item
  Complications individually reviewered where relevant.
\item
  Electronic health records accessed through GDPR compliant system and
  saved in encrypted work spaces.
\item
  All investigation results (CT imagings, blood results) in between ECMO
  run times are extracted.
\end{itemize}

\hypertarget{truths}{%
\section{3. TRUTHS}\label{truths}}

\hypertarget{assumptions}{%
\subsection{3.1. Assumptions}\label{assumptions}}

\hypertarget{assumption-1-each-patients-receive-only-1-ecmo-run}{%
\subsubsection{Assumption 1 : each patients receive only 1 ECMO
run}\label{assumption-1-each-patients-receive-only-1-ecmo-run}}

\begin{itemize}
\tightlist
\item
  verified through discussion with clinicians at regular mortality and
  morbidity meeting
\end{itemize}

\hypertarget{assumption-2-all-ecmo-pump-speed-run-times-are-accurate-with-automatic-entry-to-ehr-system}{%
\subsubsection{Assumption 2: all ECMO pump speed run times are accurate
with automatic entry to EHR
system}\label{assumption-2-all-ecmo-pump-speed-run-times-are-accurate-with-automatic-entry-to-ehr-system}}

\begin{itemize}
\tightlist
\item
  verified through daily clinical use
\item
  Thus, this time period is used as a basis to calculate other times in
  relation.
\end{itemize}

\hypertarget{assumption-3-as-mandated-legally-by-nhs-blood-and-transfusion-all-blood-products-transfusion-are-documented.}{%
\subsubsection{Assumption 3: As mandated legally by NHS Blood and
Transfusion, all blood products transfusion are
documented.}\label{assumption-3-as-mandated-legally-by-nhs-blood-and-transfusion-all-blood-products-transfusion-are-documented.}}

\begin{itemize}
\tightlist
\item
  This was used to verify and cross-check ECMO run times and related
  complications.
\end{itemize}

\hypertarget{definitions}{%
\subsection{3.2. Definitions}\label{definitions}}

\hypertarget{outcome-variables-1}{%
\subsubsection{Outcome variables}\label{outcome-variables-1}}

\hypertarget{bleeding-and-thrombotic-complications}{%
\subsubsection{Bleeding and Thrombotic
Complications}\label{bleeding-and-thrombotic-complications}}

Any relevant bleeding and thrombotic complications are recorded and
nearest most accurate time period extracted.

This does not include complications sustained through ECMO cannulation
process nor incurred through retrieval process.

In addition, relevant complications identified incidentally through
cross sectional imaging are also supplemented.

EOLIA definition of bleeding complications restrict to bleeding events
requiring transfusion of blood products.

Our definition of bleeding complication is more broad. This is to allow
analysis at granular scale.

We separately analyse blood products transfusion.

\hypertarget{ecmo-circuit-changes}{%
\subsubsection{ECMO circuit changes}\label{ecmo-circuit-changes}}

Similarly, documentation of ECMO circuit changes and rationales were
extracted.

\hypertarget{heparin-administrations}{%
\subsubsection{Heparin administrations}\label{heparin-administrations}}

All administration of heparin prescribed under different regimes were
extracted and relevant units obtained.

\begin{longtable}[]{@{}
  >{\raggedright\arraybackslash}p{(\columnwidth - 4\tabcolsep) * \real{0.1250}}
  >{\raggedright\arraybackslash}p{(\columnwidth - 4\tabcolsep) * \real{0.4722}}
  >{\raggedright\arraybackslash}p{(\columnwidth - 4\tabcolsep) * \real{0.4028}}@{}}
\toprule()
\begin{minipage}[b]{\linewidth}\raggedright
Regimes
\end{minipage} & \begin{minipage}[b]{\linewidth}\raggedright
Description
\end{minipage} & \begin{minipage}[b]{\linewidth}\raggedright
Units
\end{minipage} \\
\midrule()
\endhead
1 & Heparin & units per hour \\
2 & Heparin ECMO Xa IV Infusion & units per kilogram per hour \\
3 & Heparin IV injection & units \\
4 & Heparin RR INF 20000 IU & units \\
5 & Heparin systemic Inf 20000 Units & units per hour \\
\bottomrule()
\end{longtable}

\hypertarget{height-and-weight}{%
\subsubsection{Height and Weight}\label{height-and-weight}}

Original documentaion of height and weight on ECMO cannulation were used
due to likely later variation of patients through out their critical
illness.

\hypertarget{time-in-therapeutic-range-rosendaal-method}{%
\subsubsection{Time in Therapeutic Range (Rosendaal
method)}\label{time-in-therapeutic-range-rosendaal-method}}

Linear interpolation method by Rosendaal was used.

\$\$\begin{equation}

Time\ in\ Therapeutic\ Range\ _{Rosendaal} = \frac{Time\ in\ range}{Total\ Time }
\end{equation}\$\$

No of anticoagulation blood tests per day was calculated to detect
anomalous results. When outlier values were noted and duplicate record
with non outlier value was found, this was chosen and randomly verified
manually.

Time interval between each anticoagulation result were calculated and
TTR value for that period calculated. Total TTR of all time intervals in
between ECMO start and end times were calculated.

\hypertarget{variability-of-anticoagulation-fihns-method-of-variance-growth-rate}{%
\subsubsection{Variability of Anticoagulation (Fihn's method of Variance
Growth
rate)}\label{variability-of-anticoagulation-fihns-method-of-variance-growth-rate}}

\$\$\begin{equation}

\sigma ^{2} = \tfrac{1}{n}\tfrac{}{} \sum_{i = 1}^{n} \frac{(value_{i} - target)^{2}}{\tau _{i}}

\end{equation}\$\$

\hypertarget{data-cleaning-process}{%
\section{5. Data Cleaning Process}\label{data-cleaning-process}}

All relevant variables and time-stamps are cor

\hypertarget{exploratory-descriptive-analysis}{%
\section{6. Exploratory Descriptive
Analysis}\label{exploratory-descriptive-analysis}}

\hypertarget{analysis}{%
\section{7. Analysis}\label{analysis}}

\hypertarget{models}{%
\section{8. Models}\label{models}}

\hypertarget{model-1-time-in-therapeutic-range-ttr}{%
\subsection{8.1. Model 1 : Time-in-Therapeutic-Range
(TTR)}\label{model-1-time-in-therapeutic-range-ttr}}

\hypertarget{model-used-5-part1-analysis.r}{%
\subsubsection{Model used:
5-part1-analysis.R}\label{model-used-5-part1-analysis.r}}

\hypertarget{bmxphi}{%
\subsubsection{bmxphi}\label{bmxphi}}

\hypertarget{interpretation}{%
\subsubsection{Interpretation}\label{interpretation}}

Membership of anti-Xa group decreases by 0.95 the odds of having a
larger TTR. Each additional day on ECMO increases the odds of having a
larger TTR by 1.01. Receipt of renal replacement therapy decreases the
odds of having a large TTR by 0.6.

\hypertarget{httpsjournals.sagepub.comdoi10.11770962280217690413url_verz39.88-2003rfr_idori3arid3acrossref.orgrfr_datcr_pub0pubmed}{%
\subsection{\texorpdfstring{\url{https://journals.sagepub.com/doi/10.1177/0962280217690413?url_ver=Z39.88-2003\&rfr_id=ori\%3Arid\%3Acrossref.org\&rfr_dat=cr_pub++0pubmed}}{https://journals.sagepub.com/doi/10.1177/0962280217690413?url\_ver=Z39.88-2003\&rfr\_id=ori\%3Arid\%3Acrossref.org\&rfr\_dat=cr\_pub++0pubmed}}\label{httpsjournals.sagepub.comdoi10.11770962280217690413url_verz39.88-2003rfr_idori3arid3acrossref.orgrfr_datcr_pub0pubmed}}

for interpretation

\hypertarget{beta-regression-assumptions}{%
\subsubsection{Beta regression
assumptions}\label{beta-regression-assumptions}}

Time in Therapeutic Range is an intuitive index of measure for measuring
quality of anticoagulation.

Several definitions of TTR is available :-\\
- Traditional - Linear interpolation method by Rosendaal et al

Traditional = no of tests in range divided by total no of tests. This
does not take into account duration. Thus, Rosendaal et al has
calculated a linear interpolation method incorporating duration of
measurements.

Here, TTR(Rosendaal) is used.

TTR is a value of range between 0 to 100\%. Thus, beta-regression is
used, using \{betareg\} package.

TTR includes values including 0 and 100\%, whereas strictly
beta-regression does not include values 0 or 100\%. Thus,
data-transformation y1 = y * n-1) + 0.5 / n as per Smithson and Vekuilen
was carried out.

\begin{Shaded}
\begin{Highlighting}[]
\CommentTok{\#cor.test(dm$ttrgf,dm$ttrg)}
\end{Highlighting}
\end{Shaded}

Above findings confirmed that data transformation was appropriate.

It was hypothesised that 1. age 2. BMI 3. sex 4. apache II score 5.
monitoring group 6. renal replacement therapy 7. duration on ecmo 8.
admission period median pH value

are likely to affect ``time-in-therapeutic range''

\hypertarget{multi-variate-model-fitting}{%
\subsubsection{Multi-variate model
fitting}\label{multi-variate-model-fitting}}

Beta regression model using above 8 variables were fitted. Step-wise
variable selection was undertaken using Akaike Information Criteria.
Likelihood ratio test \{lmtest\} using function ``lrtest'' was used to
evaluate final model against a null model.

Likelihood ratio test confirms that p-value of final model is 0.000171
compared to null model

\begin{Shaded}
\begin{Highlighting}[]
\NormalTok{g }\OtherTok{\textless{}{-}} \FunctionTok{lm}\NormalTok{(Sepal.Length }\SpecialCharTok{\textasciitilde{}}\NormalTok{ . ,}\AttributeTok{data=}\NormalTok{ iris)}
\CommentTok{\#summary(g)}

\NormalTok{h }\OtherTok{\textless{}{-}}\NormalTok{ sjPlot}\SpecialCharTok{::}\FunctionTok{tab\_model}\NormalTok{(g)}

\CommentTok{\#sjPlot::print\_md}
\end{Highlighting}
\end{Shaded}

ut

\begin{Shaded}
\begin{Highlighting}[]
\CommentTok{\#lmtest::lrtest(bm0,bmx)}
\end{Highlighting}
\end{Shaded}

In this model, variables ``sex, renal replacement therapy, and duration
of ECMO'' are the only variables that have statistical significant.
Thus, a reduced model using only this 3 variables were fitted.

Reduced model and full models were compared using likelihood ratio test
and there were not statistically significant differences.

Reduced model AIC was lower than full model AIC by 5 points. As a
result, we have selected a full variable model for its ability to infer
effects of biologically plausible variables such as age, sex.

APACHE 2 score already includes pH value and APACHE 2 score was not
known to be predictive of outcome in ECMO patients. Thus, sensitivity
analysis was undertaken with both variables - APACHE 2 score and median
pH value. Models were evaluated using AIC and likelihood ratio test.
Likelihood ratio tests found that model including only APACHE score
without pH variable has the lowest chi-square value and is statistically
significant (p\textless0.0001).

Thus, this model was evaluted further.

\hypertarget{variable-dispersion}{%
\subsubsection{Variable Dispersion}\label{variable-dispersion}}

In the final model, dispersion parameter - phi coefficient- was
estimated at 2.27 and was statistically significant. The most likely
variable contributing to dispersion was `duration of ecmo'.

Thus, final model was re-fitted with the same mean equation but now with
duration of ECMO as additional regressor for the precision parameter -
phi.

\url{https://cran.r-project.org/web/packages/betareg/vignettes/betareg.pdf}

The model including of ecmo duration as regressor for the precision
parameter was statistically significant and improved a model fit,
without significant difference in estimates of other paramters. AIC of
this new model was significantly lower than model without precision
parameter.

Thus, there was a statistically significant evidence for variable
dispersion, and thus was chosen as a final model

\hypertarget{fit-assessment}{%
\subsubsection{Fit assessment}\label{fit-assessment}}

Maximum likelihood estimation was used to calculate p-values.

Model assumptions were also evaluated using diagnostic plots; and was
graphically satisfacotry for normal assumption, homeoskedasticity, and
influential observations effects.

Heteroskedasticity was also checked numerically using studentized
Bresuch-Pagan test and demonstrated no evidence of heteroskedasticity.

Multicollinearity was assessed using variable inflation factors using
\{car\} package function ``vif''. All variables have VIF score
\textless{} 2 demonstrating no evidence of multi collinearity.

Link function of logit is used.

ttrpl4

\hypertarget{model-2-variability-of-anticoagulation}{%
\subsection{8.2. Model 2 : Variability of
Anticoagulation}\label{model-2-variability-of-anticoagulation}}

\hypertarget{model-chosen-mox02}{%
\subsection{Model chosen : mox02}\label{model-chosen-mox02}}

\hypertarget{interpretation-1}{%
\subsection{Interpretation :}\label{interpretation-1}}

for 1 unit increase in lactate, our variability increases by 10.6\%
membership in anti Xa decreases variability by 86\%. Variability of
anticoagulation is a measure of quality of anticoagulation. Fihn's
method of variability was used for this study.

Lower Variability results in better control of anticoagulation

\hypertarget{choice-of-model-and-assumptions}{%
\subsubsection{Choice of model and
assumptions}\label{choice-of-model-and-assumptions}}

Variability of anticoagulation was significantly right skewed thus, a
natural logarithmic transformation was undertaken of dependent variable
and then a linear model was fitted against.

It was hypothesised that 1. age 2. BMI 3. sex 4. apache II score 5.
monitoring group 6. renal replacement therapy 7. duration on ecmo

were thought to be affecting variability of anticoagulation.

\hypertarget{multi-variate-modelling}{%
\subsubsection{Multi-variate modelling}\label{multi-variate-modelling}}

On a multi-variate modelling, as evaluated by AIC and likelihood ratio
tests, monitoring group and lactate are the only two statistically
significant variables.

Thus,a reduced model using only statistically significant model was
evaluated against a full model - there were no improvement of a reduced
model. And, due to ability to infer effects of other biologically
plausible variables, age, bmi, apache etc are included in a final model.

\hypertarget{fit-assessment-1}{%
\subsubsection{Fit assessment}\label{fit-assessment-1}}

Multiple R squared value of fitted model was 0.5 and model was
statistically significant.

Model assumptions were checked for normality, heteroskedasticity, effect
of outlying values and distribution of residuals.

Numerical check of final model using Breusch-Godfrey test confirmed
visual findings that model residuals are homoskedastic.

\hypertarget{reporting}{%
\subsubsection{Reporting}\label{reporting}}

\url{https://data.library.virginia.edu/interpreting-log-transformations-in-a-linear-model/}

\hypertarget{section-1}{%
\subsection{8.3.}\label{section-1}}

\hypertarget{section-2}{%
\subsection{8.4.}\label{section-2}}

\hypertarget{section-3}{%
\subsection{8.5.}\label{section-3}}

\hypertarget{section-4}{%
\subsection{8.6.}\label{section-4}}

\hypertarget{model-7-cumulative-dose-of-heparin}{%
\subsection{8.7. Model 7 : Cumulative Dose of
Heparin}\label{model-7-cumulative-dose-of-heparin}}

\hypertarget{model-choice}{%
\subsubsection{Model Choice}\label{model-choice}}

\begin{itemize}
\tightlist
\item
  log transformed so interpret as per above
\end{itemize}

\hypertarget{multi-variate-analysis}{%
\subsubsection{Multi-variate analysis}\label{multi-variate-analysis}}

interaction was checked ttrg:group AIC lrtest

\hypertarget{fit-assessment-2}{%
\subsubsection{Fit assessment}\label{fit-assessment-2}}

\hypertarget{reporting-1}{%
\subsubsection{Reporting}\label{reporting-1}}

\hypertarget{summary}{%
\section{9. Summary}\label{summary}}

\hypertarget{to-do}{%
\section{To Do}\label{to-do}}

\begin{enumerate}
\def\labelenumi{\arabic{enumi}.}
\tightlist
\item
  when reading the manuscript, please consider if we fulfill STROBE
  guidelines
\item
  please complete the ``units'' for measured variables in table-1(e.g.,
  CRP, creatinine)
\item
  please vote on which figures to include
\item
  two-sentence take home message and 140-character tweet
\item
  250-word abstract and 3-5 keywords
\item
  Agree on title
\item
  Suggest list of reviewers -- Nunez, Schmidt, Alain V?
\end{enumerate}

\hypertarget{next-steps}{%
\section{Next Steps}\label{next-steps}}

\begin{itemize}
\tightlist
\item
  if we(us 4) are all happy, can we circulate to other relevant authors
\item
\end{itemize}

\url{https://www.springer.com/journal/134/submission-guidelines?detailsPage=pltci_1060748\#Instructions\%20for\%20Authors_Authorship\%20principles}

\end{document}
